%%%%%%%%%%%%%%%%%%%%%%%%%%%%%%%%%%%%%%%%%
% Short Sectioned Assignment
% LaTeX Template
% Version 1.0 (5/5/12)
%
% This template has been downloaded from:
% http://www.LaTeXTemplates.com
%
% Original author:
% Frits Wenneker (http://www.howtotex.com)
%
% License:
% CC BY-NC-SA 3.0 (http://creativecommons.org/licenses/by-nc-sa/3.0/)
%
%%%%%%%%%%%%%%%%%%%%%%%%%%%%%%%%%%%%%%%%%

%----------------------------------------------------------------------------------------
%	PACKAGES AND OTHER DOCUMENT CONFIGURATIONS
%----------------------------------------------------------------------------------------

\documentclass[paper=a4, fontsize=11pt]{scrartcl} % A4 paper and 11pt font size

\usepackage[T1]{fontenc} % Use 8-bit encoding that has 256 glyphs
\usepackage{fourier} % Use the Adobe Utopia font for the document - comment this line to return to the LaTeX default
\usepackage[english]{babel} % English language/hyphenation
\usepackage{amsmath,amsfonts,amsthm} % Math packages

\usepackage{lipsum} % Used for inserting dummy 'Lorem ipsum' text into the template

\usepackage[latin1]{inputenc}

\usepackage{sectsty} % Allows customizing section commands
\allsectionsfont{\centering \normalfont\scshape} % Make all sections centered, the default font and small caps

\usepackage{fancyhdr} % Custom headers and footers
\pagestyle{fancyplain} % Makes all pages in the document conform to the custom headers and footers
\fancyhead{} % No page header - if you want one, create it in the same way as the footers below
\fancyfoot[L]{} % Empty left footer
\fancyfoot[C]{} % Empty center footer
\fancyfoot[R]{\thepage} % Page numbering for right footer
\renewcommand{\headrulewidth}{0pt} % Remove header underlines
\renewcommand{\footrulewidth}{0pt} % Remove footer underlines
\setlength{\headheight}{13.6pt} % Customize the height of the header

\numberwithin{equation}{section} % Number equations within sections (i.e. 1.1, 1.2, 2.1, 2.2 instead of 1, 2, 3, 4)
\numberwithin{figure}{section} % Number figures within sections (i.e. 1.1, 1.2, 2.1, 2.2 instead of 1, 2, 3, 4)
\numberwithin{table}{section} % Number tables within sections (i.e. 1.1, 1.2, 2.1, 2.2 instead of 1, 2, 3, 4)

\setlength\parindent{0pt} % Removes all indentation from paragraphs - comment this line for an assignment with lots of text

%----------------------------------------------------------------------------------------
%	TITLE SECTION
%----------------------------------------------------------------------------------------

\newcommand{\horrule}[1]{\rule{\linewidth}{#1}} % Create horizontal rule command with 1 argument of height

\title{	
\normalfont \normalsize 
\textsc{University of Bologna - Campus of Cesena - Second Faculty of
Engineering}
\\
[25pt]
% Your university, school and/or department name(s)
\horrule{0.5pt} \\[0.4cm] % Thin top horizontal rule
\huge A music recommendation system using association rules
\\
% The assignment title
\horrule{2pt} \\[0.5cm] % Thick bottom horizontal rule
}

\author{Fabbri Francesco} % Your name

\date{\normalsize\today} % Today's date or a custom date

\begin{document}

\maketitle % Print the title

%----------------------------------------------------------------------------------------
%	PROBLEM 1
%----------------------------------------------------------------------------------------

\section{Introduction}
Scopo del lavoro\ldots\\
Nella prima parte\ldots\\
Nella seconda parte\ldots\\

\section{Background}
In \cite{AdomaviciusTuzhilin} Adomavicius et al. presents an overview of
recommender systems and describes the current generation of recommendation
methods and state-of-art techniques that can be classified into the following
three main categories: content-based, collaborative, and hybrid approaches.
Since the appearance of the first papers on collaborative filtering in the
mid-1990s the interest in this area still remains high because it constitutes a
problem-rich research area and because of the abundance of practical
applications. Examples of such applications include recommending books, CDs,
movies, news, and other products from e-commerce websites. The roots of
recommender systems can be traced back to the extensive work in cognitive
science, approximation theory, information retrieval and forecasting theories.
In the most common formulation, the recommendation problem is reduced to the
problem of estimating ratings for the items that have not been seen by a user,
this estimation is usually based on the ratings given by this user to other
items and on some other information. Under these assumptions we can recommend to
the user the items with the highest estimated ratings.\\
Formally let \textit{C} be the set of all users and let \textit{S} be the set of
all possible items that can be recommended, such as books, movies, or
restaurants. Both the spaces \textit{C} of users and \textit{S} of possible
items can be very large. Let \textit{u} be a utility function that measures the
usefulness of item \textit{s} to user \textit{c}, $u : C \times S \rightarrow
R$, where \textit{R} is a totally ordered set, usually represented by a rating,
which indicates how a particular user liked a particular item.
Then, for each user $c \in C$, we want to choose such item $s' \in S$ that
maximizes the user's utility.
$$\forall c \in C, \quad s'_c = arg \ max_{s \in S} \ u(c,s).$$
The central problem of recommender systems lies in that utility u is usually not
defined on the whole $C \times S$ space, but only on some subset of it,
furthermore the recommendation engine should be able to estimate (predict) the
ratings of the nonrated item/user combinations. Extrapolations from known to
unknown ratings are usually done by specifying heuristics that define the
utility function and empirically validating its performance or estimating the
utility function that optimizes certain performance criterion, such as the mean
square error. Once the unknown ratings are estimated, actual recommendations of
an item to a user are made by selecting the highest rating among all the
estimated ratings for that user, or alternatively, we can recommend the N best
items to a user or a set of users to an item. Therefore recommender systems are
usually classified into the following categories, based on how recommendations
are made:
\begin{itemize}
  \item \textit{Content-based recommendations}: the user will be
recommended items similar to the ones the user preferred in the past. The
utility $u(c,s)$ of item $s$ for user $c$ is estimated based on the utilities
$u(c,s_i)$ assigned by user $c$ to items $s_i \in S$ that are ``similar'' to
item $s$. This approach tries to understand the commonalities among the movies
user $c$ has rated highly in the past. Common techniques for content-based
recommendation are heuristics, Bayesian classifiers and various machine learning
techniques, including clustering, decision trees, and artificial neural
networks. These techniques differ from information retrieval-based approaches in
that they calculate utility predictions based not on a heuristic formula, such
as a cosine similarity measure, but rather are based on a model learned from the
underlying data using statistical learning and machine learning techniques.
Content-based techniques are limited by the features that are explicitly
associated with the objects that these systems recommend. While information
retrieval techniques work well in extracting features from text documents, some
other domains (i.e. multimedia data) have an inherent problem with automatic
feature extraction. Another limitation is the \textit{overspecialization} and
consists in the fact that the system can only recommend items that score highly
against a user's profile, the user is limited to being recommended items that
are similar to those already rated. This problem, which has also been studied in
other domains, is often addressed by introducing some randomness. For example,
the use of genetic algorithms has been proposed as a possible solution. Ideally,
the user should be presented with a range of options and not with a homogeneous
set of alternatives. Finally another issue concerns new users, in fact user has
to rate a sufficient number of items before a content-based recommender system
can really understand the user's preferences and present the user with reliable
recommendations.

  \item \textit{Collaborative recommendations}: the user will be
recommended items that people with similar tastes and preferences liked in the
past, in particular they try to predict the utility of items for a particular
user based on the items previously rated by other users. Formally, the utility
$u(c,s)$ of item $s$ for user $c$ is estimated based on the utilities $u(c_j,s)$
assigned to item $s$ by those users $c_j \in C$ who are ``similar'' to user $c$.
Algorithms for collaborative recommendations can be grouped into two general
classes:
\textit{memory-based (or heuristic-based)} and
\textit{model-based}.\\
\textit{Memory-based} algorithms essentially are heuristics that
make rating predictions based on the entire collection of previously rated items by the users. That is, the value of the unknown rating $r_{c,s}$ for user $c$ and item $s$ is usually
computed as an aggregate of the ratings of some other (usually, the $N$ most
similar) users $sim(c,c')$ is essentially a distance measure and is used as a
weight. Various approaches have been used to compute the similarity $sim(c,c')$
between users in collaborative recommender systems. In most of these approaches,
the similarity between two users is based on their ratings of items that both
users have rated. The two most popular approaches are correlation and
cosine-based.\\
\textit{Model-based} algorithms use instead the collection of ratings to learn a
model, which is then used to make rating predictions. Are based on the
probability that user $c$ will give a particular rating to item $s$ given that
user's ratings of the previously rated items. To estimate this probability, was
proposed two alternative probabilistic models: cluster models and Bayesian
networks.\\
The pure collaborative recommender systems do not have some of the shortcomings
that content-based systems have. In particular they can deal with any kind of
content and recommend any items, even the ones that are dissimilar to those seen
in the past. Overstay the new user limitation since in order to make accurate
recommendations, the system must first learn the user's preferences from the
ratings that the user gives. In addiction collaborative approaches soffer for
new item limitations since until the new item is rated by a substantial number
of users, the recommender system would not be able to recommend it. Finally
another limitation is given by the \textit{sparsity}, since the number of
ratings already obtained is usually very small compared to the number of ratings
that need to be predicted. Also, the success of the collaborative recommender
system depends on the availability of a critical mass of users.

  \item \textit{Hybrid approaches}: these methods combine collaborative
and content-based methods for the purpose to avoid certain limitations of both.
Different ways was proposed in literature, most common approaches are 1)
implementing collaborative and content-based methods separately and combining
their predictions, 2) incorporating some content-based characteristics into a
collaborative approach, 3) incorporating some collaborative characteristics into
a content-based approach, and 4) constructing a general unifying model that
incorporates both content-based and collaborative characteristics. Moreover was
empirically demonstrated that the hybrid methods can provide more accurate
recommendations than pure approaches.
\end{itemize}

In \cite{ChenChen} Chen et al. propose a music recommendation service based on
content-based, collaborative and statistics-based approaches. The music objects
of MIDI format are analyzed. Based on the perceptual properties of music
objects, six features are extracted from each representative track such as mean
and standard deviation of pitch values, pitch density, pitch entropy, tempo
degree and loudness. From the trials results that the content-based approach
give more accurate recommendation than collaborative and statistic approach.

\section{Design}
TODO\\

\subsection{Business understanding}
TODO\\

\subsection{Data understanding}
TODO\\

\subsection{Data preparation}
TODO\\

\subsection{Modeling}
TODO\\

\subsection{Evaluation}
TODO\\

\subsection{Deployment}
TODO\\

\section{Conclusions}
TODO\\

\newpage

\begin{thebibliography}{9}

\bibitem{AdomaviciusTuzhilin}
 Gediminas Adomavicius, Alexander Tuzhilin\\
 \emph{``Toward the Next Generation of Recommender Systems - A Survey of the State-of-the-Art and Possible Extensions''}\\
 \emph{IEEE TRANSACTIONS ON KNOWLEDGE AND DATA ENGINEERING, VOL. 17, NO. 6, JUNE 2005}.\\

\bibitem{KimKim}
 Choonho Kim, Juntae Kim\\
 \emph{``A Recommendation Algorithm Using Multi-Level Association Rules''}\\
 \emph{IEEE/WIC International Conference on Web Intelligence (WI'03), 2003}.\\
    
\bibitem{ChenChen}
 Hung-Chen Chen, Arbee L.P. Chen\\
 \emph{``A music recommendation system based on music data grouping and user interests''}\\
 \emph{CIKM'OI, 2001}.\\
   
\bibitem{LeeChoKim}
 Seok Kee Lee, Yoon Ho Cho, Soung Hie Kim\\
 \emph{``Collaborative filtering with ordinal scale-based implicit ratings for mobile music recommendations''}\\
 \emph{Journal Information Sciences: an International Journal archive, June 2010}.\\

\bibitem{Playlist.com}
 \emph{http://www.playlist.com}\\

\end{thebibliography}

\end{document}